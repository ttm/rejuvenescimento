\documentclass[a4paper]{article}

\usepackage{hyperref}
\usepackage[portuguese]{babel}
\usepackage[utf8]{inputenc}
\usepackage{amsmath}
\usepackage{graphicx}
\usepackage[colorinlistoftodos]{todonotes}

\usepackage{tocloft}
\addtocontents{toc}{\cftpagenumbersoff{section}}


\title{Psicofísica do rejuvenescimento}
\author{Jesus de Nazaré e Paulo de Tarso}

\date{\today, versão 0.01- alfa}

\usepackage{etoolbox}

\makeatletter
\pretocmd{\chapter}{\addtocontents{toc}{\protect\addvspace{15\p@}}}{}{}
\pretocmd{\section}{\addtocontents{toc}{\protect\vspace{-3mm}}}{}{}
\makeatother


\begin{document}
\maketitle



\begin{abstract}
Explicitação formal dos mecanismos psicofísicos responsáveis pelo empoderamento do indivíduo pelo próprio inconsciente.
\end{abstract}

\tableofcontents

\section{Ponto de vista e propósito deste texto}
O sistema nervoso humano possui capacidades para processamento de informação
que o inconsciente usa constantemente.
Por exemplo, $\approx$ um trilhão e trilhões
fótons atingem cada olho a cada segundo, são codificados, processados e transportados
por milhões de neurônios do nervo ótico e, no sistema nervoso central, cada objeto é identificado
e contextualizado para usos adequados. Isso automaticamente o tempo todo.
O mesmo pode ser observado para a audição, outros sentidos, atividades sociais, etc.

Considerada esta capacidade, é evidente que os erros cometidos são programados.
Um livro ou guarda-chuva esquecido, é esquecido de propósito, seja a ação
acessada pelo consciente ou não.
Para este texto, consideramos dois tipos de vantagens:
a vantagem que empodera o indivíduo e a vantagem que sabota o indivíduo.
Por que há vantagem na autossabotagem? Porque assim a espécie pode
atribuir espaço para cada indivíduo conforme o quão benéfico forem para a perpetuação
(indivíduo, comunidades e espécie como um todo).

Este texto revela este mecanismo de (i)limitação das capacidades individuais de maneira concisa e com formalismos.
Em especial, são apresentados recursos que facilitam a conquista da realidade
(i.e. iluminação) e seu papel para a estratégia humana de sobrevivência. 




\section{Energias interna e externa}
\section{Despersonificação}
\section{Ubiquidade das estruturas}
\section{Aproveitamento eficiente}



\end{document}

